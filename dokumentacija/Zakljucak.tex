\chapter{Zaključak i budući rad}
		\begin{comment}
			content..\textbf{\textit{dio 2. revizije}}\\
			
			\textit{U ovom poglavlju potrebno je napisati osvrt na vrijeme izrade projektnog zadatka, koji su tehnički izazovi prepoznati, jesu li riješeni ili kako bi mogli biti riješeni, koja su znanja stečena pri izradi projekta, koja bi znanja bila posebno potrebna za brže i kvalitetnije ostvarenje projekta i koje bi bile perspektive za nastavak rada u projektnoj grupi.}
			
			\textit{Potrebno je točno popisati funkcionalnosti koje nisu implementirane u ostvarenoj aplikaciji.}.
		\end{comment}
		
		Projekt kojim smo se bavili tijekom semestra bio je izrada web aplikacije 
		za rezerviranje termina u praonici rublja koja bi
		prvenstveno olakšala taj proces studentima, ali i zaposlenicima iste. Aplikacija ima potencijal primjene  i na druge sustave.
		Nakon tromjesečnog
		timskog rada i razvoja aplikacije ostvarili smo zadani cilj. Izvedba projekta
		odvijala se kroz dvije faze.
		
		U prvoj fazi projekta naglasak je bio na okupljanju tima, 
		njegovom upoznavanju, analiziranju sposobnosti i okvirnoj raspodjeli
		budućih poslova. Isto tako, većina se vremena posvetila
		osmišljavanju funkcionalnosti, analiziranju zahtjeva te 
		dokumentiranju istih. Detaljna dokumentacija u koju smo uložili mnogo truda
		i vremena bila je izvrstan temelj za drugu fazu izvedbe projekta. 
		Funkcionalni zahtjevi, obrasci uporabe, sekvencijski dijagrami, model baze
		podataka i dijagram razreda bili su veoma koristan alat kojim smo rješavali
		implementacijske nedoumice. 
		
		Druga faza projekta sadržavala je u najvećem dijelu programiranje i
		implementaciju dokumentirane web aplikacije. Ta faza zahtjevala je 
		poseban samostalni angažman članova tima koji se dotada nisu susreli
		s tehnologijama koje smo odlučili koristiti, ali i angažman 
		članova tima koji su s tim tehnologijama bili upoznati kako bi
		pomogli kolegama da ih što uspješnije savladaju. Tijekom tog vremena
		vladao je visok intenzitet rada i odlična timska kohezija. Isto tako, 
		bilo je potrebno napisati preostalu dokumentaciju i izraditi UML dijagrame.
		
		Pri izradi projekta mnogo toga smo naučili, od korištenja novih tehnologija,
		alata, usavršavanja programskih jezika, izrade dokumentacije
		do važnosti i koristi koje nam pruža timski rad, kolegijalni i dobri
		timski odnosi, organiziran i marljiv voditelj tima koji koordinira
		svim aktivnostima, ali i vrijedni članovi
		koji prihvaćaju sugestiju te teže napretku u svakom smislu.
		Također smo naučili vrijednost kontinuiranog rada koji nam je olakšao
		da sve zadatke obavimo na vrijeme i bez panike. Izuzetno smo zadovoljni
		izrađenom web aplikacijom, ali smo isto tako svjesni da ima dosta mjesta
		za napredak i usavršavanje iste. Jedna od mogućnosti bi bila proširenje 
		aplikacije za hotelske sustave ili samoposlužne praonice, ali i izrada mobilne 
		aplikacije.
		\eject 