\chapter{Dnevnik promjena dokumentacije}
		
		\textbf{\textit{Kontinuirano osvježavanje}}\\
				
		
		\begin{longtabu} to \textwidth {|X[2, l]|X[13, l]|X[5, l]|X[4, l]|}
			\hline \multicolumn{1}{|l|}{\textbf{Rev.}}	& \multicolumn{1}{l|}{\textbf{Opis promjene/dodatka}} & \multicolumn{1}{|l|}{\textbf{Autori}} & \multicolumn{1}{l|}{\textbf{Datum}} \\[3pt] \hline
			\endfirsthead
			
			\hline \multicolumn{1}{|l|}{\textbf{Rev.}}	& \multicolumn{1}{l|}{\textbf{Opis promjene/dodatka}} & \multicolumn{1}{|l|}{\textbf{Autori}} & \multicolumn{1}{l|}{\textbf{Datum}} \\[3pt] \hline
			\endhead
			
			\hline 
			\endlastfoot
			
			0.1 & Napravljen predložak.	& D. Grgić & 20.10.2020. 		\\[3pt] \hline 
			0.2	& Promijenjeni dijelovi predloška. & J. Grgić & 21.10.2020. 	\\[3pt] \hline 
			0.3 & Prepisan opis projektnog zadatka iz word dokumenta koji je sastavljen ranije radi prijave vlastite teme. & J. Grgić & 22.10.2020. 		\\[3pt] \hline
			0.4 & Dodani dionici u poglavlju 3, navedeni aktori i funkcionalni zahtjevi za neregistrirane/neprijavljene korisnike. Napisani obrasci uporabe za neregistrirane/neprijavljene korisnike.	& J. Grgić & 23.10.2020. 		\\[3pt]
		    \hline 
		    0.5 & Uređen dnevnik sastajanja i podjela poslova u grupi.	& J. Grgić & 23.10.2020. 		\\[3pt] \hline
		    0.6 & Dodani funkcionalni zahtjevi za Administratora. & I. Joskić & 23.10.2020. \\[3pt] \hline
			0.7 & Dodani funkcionalni zahtjevi za Korisnika.  & B. Spiegl & 23.10.2020. \\[3pt] \hline
		    0.8 & Dodani obrasci uporabe za Administratora. & I. Joskić & 23.10.2020. \\[3pt] \hline
		    0.9 & Navedeni funkcionalni zahtjevi za Zaposlenika. & M. Dragošević & 24.10.2020. \\[3pt] \hline
		    0.10 & Dodani obrasci uporabe za Zaposlenika. & M. Dragošević & 24.10.2020. \\[3pt] \hline
		    0.11 & Dodani opisi tablica u bazi podataka. & D. Šmigovec & 25.10.2020. \\[3pt] \hline
		    0.12 & Promjena baze podataka. & D. Šmigovec & 25.10.2020. \\[3pt] \hline
			0.13 & Dodan 3. sastanak u dnevnik sastajanja & J. Grgić & 26.10.2020. \\[3pt] \hline
			0.14 & Dodani ostali zahtjevi. & D. Grgić & 26.10.2020. \\[3pt] \hline
			0.15 & Dodano još obrazaca uporabe za Administratora. & I. Joskić & 26.10.2020. \\[3pt] \hline
			0.16 & Dodano još obrazaca uporabe za Registriranog korisnika i modificirani funkcionalni zahtjevi zajedno sa odgovarajućim obrascima upotrebe. & B. Spiegl & 29.10.2020. \\[3pt] \hline
			0.17 & Dodan obrazac uporabe za plaćanje i prvi dijagram obrasca uporabe za registrirane i neregistrirane korisnike. & J. Grgić & 31.10.2020. \\[3pt] \hline
			0.18 & Dodani sekvencijski dijagrami za UC1 i UC11. & I. Joskić & 31.10.2020. \\[3pt] \hline
			0.19 & Dodani sekvencijski dijagrami za UC3 i UC4. & D. Šmigovec & 31.10.2020. \\[3pt] \hline
			0.20 & Dodani opisi za sekvencijske dijagrame \ref{fig:skvDReg} i \ref{fig:skvDOglas}. & I. Joskić & 1.11.2020. \\[3pt] \hline
			0.21 & Dodani dijagrami obrazaca uporabe za zaposlenika i administratora. & B. Spiegl & 1.11.2020. \\[3pt] \hline
			0.22 & Dodan opis i slika arhitekture sustava. & D. Grgić & 1.11.2020. \\[3pt] \hline
			0.23 & Popravljen dijagram obrazaca uporabe za registriranog i neregistriranog korisnika. & J. Grgić & 1.11.2020. \\[3pt] \hline
			0.24 & Dodana shema Django baze. & D. Šmigovec & 1.11.2020. \\[3pt] \hline
			0.25 & Popravljen UC2 i dijagram obrasca uporabe 1. Dodana literatura. & J. Grgić & 1.11.2020. \\[3pt] \hline
			0.26 & Dodani razredni dijagrami. & D. Grgić & 7.11.2020. \\[3pt] \hline
			0.27 & Nadopunjeni opisi tablica i promjena atributa na engleski. & D. Šmigovec & 7.11.2020. \\[3pt] \hline
			0.27 & Nadopunjeni dijagrami razreda prema predpostavljenoj zavšrnoj implementaciji. & D. Grgić 9.11.2020. \\[3pt] \hline			
			

			
		\end{longtabu}
	
	
		\textit{Moraju postojati glavne revizije dokumenata 1.0 i 2.0 na kraju prvog i drugog ciklusa. Između tih revizija mogu postojati manje revizije već prema tome kako se dokument bude nadopunjavao. Očekuje se da nakon svake značajnije promjene (dodatka, izmjene, uklanjanja dijelova teksta i popratnih grafičkih sadržaja) dokumenta se to zabilježi kao revizija. Npr., revizije unutar prvog ciklusa će imati oznake 0.1, 0.2, …, 0.9, 0.10, 0.11.. sve do konačne revizije prvog ciklusa 1.0. U drugom ciklusu se nastavlja s revizijama 1.1, 1.2, itd.}